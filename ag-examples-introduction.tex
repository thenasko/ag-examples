\chapter{Introduction}

\section{Motivation}

TODO

\section{Philosophy}

TODO

\section{Guidelines}

Below is a set of guidelines which were used in the compilation of this document. They do not reflect any absolute practice in typesetting, but merely the combination of my own customs and some rational decisions which I took to standardize various parts of the document.
\begin{itemize}
\item
  When using material from a book or article add a reference to it. If the discussion uses results or definitions which are not well-know, it could be beneficial to note these. Do not hesitate to add your own relevant observations as these could facilitate the learning process.
\item
  When a source uses notation or language which is slightly old, it is better to update that. Best care should be taken to ensure notation is consistent throughout all sections.
\item
  Use as many of the predefined customizations as possible.
\item
  Avoid the use of ``one-to-one'' and ``onto''. Instead replace these with ``injective'' and ``surjective'' respectively.
\item Use $\setminus$ for the difference of sets instead of $-$.
\item Use \texttt{\textbackslash cn} (without the customizations \texttt{\textbackslash colon}) for colons in functions, that is, visually $f \cn X \rarr Y$ looks better than $f : X \rarr Y$ (note the difference in spacing before the colon).
\item Be consistent with wording and spelling. On a similar note, hyphenate ``non'' constructions such as ``non-constant'' and ``non-vanishing''.
\end{itemize}


%%% Local Variables: 
%%% mode: latex
%%% TeX-master: "ag-examples"
%%% End: 
