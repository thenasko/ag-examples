\chapter{Introduction}

\section{Motivation}

Examples in algebraic geometry is an open source project whose content is succinctly described by its name.

There are many beautiful, instructional, and inspiring examples in algebraic geometry. On the other hand, the foundations of the subject are rather technical, which makes it more difficult for students to gain geometric intuition. Our aim of this project is to assemble a collection of "examples", short narratives of various phenomena, and to organize them thematically.

Learning algebraic geometry can be a lengthy and arduous process. Everyone has a story in which they are working hard trying grasp a complex idea and then they reach a moment of clarity. Afterwards, the concept seems natural and readily approachable. While these experiences can be very personal, they are often catalyzed by a good example. We would like to help future students by collecting and cataloging these nuggets of information.

\section{Guidelines}

Below is a set of guidelines which are used throughout of this document. They do not reflect any absolute practice in typesetting, but merely the combination of personal customs and some rational decisions aimed at standardizing various parts of the document.
\begin{itemize}
\item
  When using material from a book or article add a reference to it. If the discussion uses results or definitions which are not well-know, it could be beneficial to note these. Do not hesitate to add your own relevant observations as these could facilitate the learning process.
\item
  When a source uses notation or language which is slightly old, it is better to update that. Best care should be taken to ensure notation is consistent throughout all sections.
\item
  Use as many of the predefined customizations as possible.
\item
  Referring to maps as ``one-to-one'' and ``onto'' can be very confusing. Replace these with ``injective'' and ``surjective'' respectively.
\item Use $\setminus$ for the difference of sets instead of $-$.
\item Use \texttt{\textbackslash cn} (without the customizations \texttt{\textbackslash colon}) for colons in functions, that is, visually $f \cn X \rarr Y$ looks better than $f : X \rarr Y$ (note the difference in spacing before the colon).
\item Be consistent with wording and spelling. On a similar note, hyphenate ``non'' constructions such as ``non-constant'' and ``non-vanishing''.
\end{itemize}


%%% Local Variables: 
%%% mode: latex
%%% TeX-master: "ag-examples"
%%% End: 
