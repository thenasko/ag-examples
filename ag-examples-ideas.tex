\chapter{Ideas}

\begin{itemize}
\item Non-projective threefold (Hironaka's example)
\item Non-projective surface (singular, see Vakil)
\item A curve with a non-smoothable singularity
\item A curve with non-planar singularity (maybe give an example with arbitrary dimensional tangent space)
\item Examples of plane singularities; it is not hard to write down specific examples with any torus knot; give an example with a cable of a torus knot
\item Lines on a quadric in $\Pbb^3$ and $\Pbb^4$
\item Variety whose effective cone is not polygonal
\item Nef divisor which is not effective
\item Several examples of semistable reduction for curves
\item A complex manifold which is not algebraic (e.g., a torus) (see Shafarevich for more ideas)
\item Monodromy of 27 lines on a cubic surface
\item Monodromy of 2 rulings on smooth quadric surface degenerating to a cone
\item Classify length 2 and 3 schemes (see Geometry of Schemes)
\item Symmetric products versus Hilbert schemes of points
\item Explain why symmetric products of curves are smooth, but this is not true for higher dimensions
\item Examples which illustrate behavior specific to schemes
\item Examples which illustrate behavior specific to stacks (Deligne-Mumford, Artin) (see Introduction in Knutson, Algebraic spaces)
\item Interesting things about curves
\item Interesting things about hypersurfaces
\item Triple, quadruple covers, etc
\item Cyclic covers
\item Explicit examples of resolutions of singularities, especially emphasizing the differences between the classes of singularities coming from MMP
\item Explain why hypersurfaces of small degree are uniruled
\item Explain why locally free in the Zariski and \'etale topologies is equivalent (related to Hilbert theorem 90)
\end{itemize}

\section{Non-planar curve singularities}

It is easy to construct a variety of dimension 1 whose tangent space at a given point has arbitrary dimension $n \geq 1$. For example, consider the union of $n$ lines in $\Abb^n$, all passing through a given point $p \in \Abb^n$, whose tangent spaces span the ambient tangent space $T_p \Abb^n$. To be concrete, let $x_1, \dots, x_n$ be global coordinates on $\Abb^n$; take $I_i = (x_1, \dots, \what{x}_i, \dots, x_n) \subset k[x_1, \dots, x_n]$, $L_i = V(I_i)$, and $p = (0, \dots, 0)$. At the origin $p$, the variety $C = \bigcup_{i=1}^n L_i = V(I_1 \cdots I_n)$ has tangent space
\[
T_p C =
\sum_{i=1}^n T_p L_i =
\sum_{i=1}^n k \left\langle \frac{\d}{\d x_i} \right\rangle =
k \left\langle \frac{\d}{\d x_1}, \dots, \frac{\d}{\d x_n} \right\rangle =
T_p \Abb^n
\]
of dimension $n$.

TODO: Diagram

It is not too much harder to come up with an irreducible curve whose tangent space has arbitrary dimension.



TODO


\section{Planar curve singularities}

TODO


\section{Vector bundle which does not extend}

We will exhibit an example of a vector bundle $E$ on an open subset $U$ of a variety $X$ which does not extend to all of $X$.

Let $E'$ be a vector bundle on $\Pbb^2$ which does not split into a direct sum of line bundles. To be more concrete, take the tangent bundle $E' = T_{\Pbb^2}$. Consider $\Abb^3$ and the open subset $\Abb^3 \setminus \{0\}$. Let $\pi \cn \Abb^3 \setminus \{0\} \rarr \Pbb^2$ be the natural projection map. We claim the vector bundle $E = \pi^\ast E'$ does not extend to $\Abb^3$. If it did, it has to be trivial (TODO: Reference to Quillen-Suslin). In particular, it would split which is not the case. (TODO: Elucidate the connection between $\Pbb^2$ and $\Abb^3 \setminus \{0\}$ better here.)

Note that $E$ extends to a vector bundle on $\Bl_{0} \Abb^3$ since the projection map extends to a map $\Bl_0 \Abb^3 \rarr \Pbb^2$.

\begin{remark}
  While we are providing a sort of counterexample, it would be interesting to investigate this problem more closely. For example, given a coherent sheaf $E$ on $X$ which is a vector bundle on $U$, there always exists a blowup $X'$ of $X$ with center in $X \setminus U$ such that $E|_U$ extends to a vector bundle over $X'$ \cite[Theorem 4.1]{Raynaud-Flat-modules}. Is there a way to carry this process in a canonical fashion?
\end{remark}


\section{An explanation why $T_{\mathbb{P}^2}$ does not split}

There are (at least) two heavy\--machinery reasons for this. One is that $T_{\mathbb{P}^2}$ is a stable vector bundle on $\mathbb{P}^2$, and stable bundles are simple (i.e.\ their endomorphisms are all homotheties) and therefore do not split. This last implication is easy to explain: if $E$ splits as ${E=E_1\oplus E_2}$, then it admits for instance the endomorphism ${(x_1,x_2)\mapsto(x_1,0)}$, which is not a homothety.

The other is Horrocks criterion for splitting of vector bundles on the projective space \cite{Horrocks-Punctured-spectrum}. A vector bundle $E$ on $\mathbb{P}^n$ splits if and only if it has no intermediate cohomology, i.e.\ ${\H^i\big(\mathbb{P}^n,E(a)\big)=0}$ for all ${i\notin\{0,n\}}$ and all ${a\in\mathbb{Z}}$. If we consider the dual of the Euler sequence and twist it by ${\mathcal{O}_{\mathbb{P}^2}(-3)}$ we obtain the sequence
\[\xymatrix{0\ar[r] &\mathcal{O}_{\mathbb{P}^2}(-3)\ar[r] 
  &\mathcal{O}_{\mathbb{P}^2}(-2)^2\ar[r]
  &T_{\mathbb{P}^2}(-3)\ar[r] &0.}\]
We can take cohomology and since
\[
{h^1\big(\mathcal{O}_{\mathbb{P}^2}(-2)^2\big)=h^2\big(\mathcal{O}_{\mathbb{P}^2}(-2)^2\big)=0},
\]
we get
\[
{\H^1\big(T_{\mathbb{P}^2}(-3)\big)\cong \H^2\big(\mathcal{O}_{\mathbb{P}^2}(-3)\big)\cong k}.
\]

For a definition of stable bundles and the result that they are simple, see chapter~II, section 1 in \cite{OSS-Vector-bundles}; for Horrocks criterion, see chapter~I, section 2.3.  


\section{Double covers}

Let $X$ be a variety (smooth over an algebraically closed field $k$ of characteristic $0$). We would like to describe the set of (potentially ramified) double covers $\pi \cn Y \rarr X$.

Start by observing these correspond naturally to flat, locally free rank 2 $\Oc_X$-algebras $\Ac$. Given a cover $\pi \cn Y \rarr X$, we construct such a sheaf by $\Ac = \pi_\ast \Oc_Y$. Conversely, given a sheaf of algebras, taking $\Spec$ furnishes a cover. Therefore, it suffices to study such $\Oc_X$-algebras.

The unit in such an $\Oc_X$-algebra $\Ac$, gives an embedding $i \cn \Oc_X \rarr \Ac$. This map admits a splitting
\[
s = \frac{1}{2} \Tr \cn \Ac \rarr \Oc_X.
\]
It is best to illustrate the trace map when $X = \Spec k$ and $\Ac = \wtilde{A}$ is a length 2 algebra over $k$. In simple terms, $A$ is a dimension $2$ vector space over $k$. Multiplication by an element $x \in A$ induces an endomorphism $x \cdot \cn A \rarr A$. The trace map $\Tr \cn A \rarr k$ is defined as the trace of this endomorphism, namely,
\[
\Tr(x) = \Tr( x \cdot \cn A \rarr A ).
\]
We are using the fact $\Tr(1_A) = \dim_k A = 2$, so $s = 1/2 \Tr$ splits the inclusion $k \hrarr A$. Note that $1/n \Tr$ defines a splitting for any length $n$ algebra.

Back to a general base $X$, we have constructed a split short exact sequence
\[\xymatrix{
  0 \ar[r] &
  \Oc_X \ar[r]^-{i} &
  \Ac \ar@/^/[l]^-{s} \ar[r] &
  \wtilde{\Ac} \ar[r] &
  0.
}\]
Since $\Ac = \Oc_X \oplus \wtilde{\Ac}$ is locally free of rank $2$, it follows that $L = \wtilde{\Ac}$ is a line bundle. The multiplication map $\mu \cn \Ac \otimes \Ac \rarr \Ac$, can be written as a quadruple of morphisms
\[
\Oc_X \otimes \Oc_X \rarr \Ac, \qquad
\Oc_X \otimes L \rarr \Ac, \qquad
L \otimes \Oc_X \rarr \Ac, \qquad
L \otimes L \rarr \Ac.
\]
The first three of these are uniquely determined by the fact $\Oc_X \subset \Ac$ is the subsheaf spanned by the unit. The last one can further be decomposed into a pair of morphisms
\[
L \otimes L \rarr \Oc_X, \qquad
L \otimes L \rarr L.
\]
We claim that working with double covers forces $L \otimes L \rarr L$ to be the zero map. Again, it is best to see this by looking one point at a time.

There are two isomorphism classes of length 2 algebras, namely $A_1 = k \x k$ and $A_2 = k[\epsilon]/(\epsilon^2)$. The subspaces of trace-free elements (analogues of $L$ above) in each are given by
\[
L_1 = k \cdot (1,-1) \subset A_1, \qquad
L_2 = k \cdot \epsilon \subset A_2.
\]
In the first case, the product $(1,-1) \cdot (1,-1) = (1,1)$ has projection $0$ in the trace-free part. In the second case $\epsilon \cdot \epsilon = 0$, so there is no need to project. In each of the two cases, the map $L_i \otimes L_i \rarr L_i$ is zero. Note that the map $L_i \otimes L_i \rarr k$ is zero only in the second case. Geometrically, $A_1$ corresponds to reduced fiber of two points, while $A_2$ to a point of ramification. We will later use this observation to detect the ramification locus.

Returning to a general base, the multiplication map $\Ac \otimes \Ac \rarr \Ac$ is determined by a map $\sigma \cn L \otimes L \rarr \Oc_X$. Conversely, it is not hard to check that any map $\sigma \cn L \otimes L \rarr \Oc_X$ can be augmented to give a commutative, associative $\Oc_X$-algebra structure on $\Ac = \Oc_X \oplus L$, the isomorphism type of the algebra depends only on $\sigma$ up to scaling. Finally, we can take duals to get
\[
\sigma^\vee \cn \Oc_X \rarr L^{-2}.
\]
Up to scaling this is the same as a divisor $D$ in the linear system $|L^{-2}|$. Note that $D$ is the vanishing locus of $\sigma^\vee$ and $\sigma$, hence also the ramification locus of the corresponding double cover $\Spec(\Oc_X \otimes L) \rarr X$. We can replace $L$ with $L^{-1}$ for convenience to obtain the result as it is commonly stated.

\begin{proposition}
  A pair $(L,D)$ consisting of a line bundle $L \in \Pic(X)$ together with a divisor $D \in |L^2|$ determine a double cover, and conversely. The ramification divisor of the cover corresponding to a pair $(L,D)$ is $D$.
\end{proposition}

We can use this result to derive a useful corollary.

\begin{corollary}
  Unramified double covers of $X$ are in a correspondence with 2-torsion elements of $\Pic(X)$.
\end{corollary}

TODO: The following paragraph required revision.

Given a pair $(L,D)$ as above, there is an alternative geometric construction of the associated double cover. Start by using $D$ to produce a map $f \cn X \rarr \Pbb^1$ such that $D = [f^{-1}(0)] - [f^{-1}(\infty)]$. It is easy to construct a double cover $C \rarr \Pbb^1$ with unique ramification over $0$ and $\infty$. The fiber product $Y = X \x_\Pbb^2 C \rarr X$ is the desired double cover.
\[\xymatrix{
  Y = X \x_{\Pbb^1} C \ar[r] \ar[d] & C \ar[d] \\
  X \ar[r] & \Pbb^1
}\]


\section{The scroll $\Fbb(a_1, \dots, a_n)$ as a quotient}

It is customary to construct the scroll $\Fbb(a_1, \dots, a_n)$ as the projectivization $\Pbb E$ of the vector bundle
\[
E = \Oc(a_1) \oplus \cdots \oplus \Oc(a_n)
\]
over $\Pbb^1$. Alternatively, it is also possible to construct this space as the quotient of $(\Abb^2 \setminus 0) \x (\Abb^n \setminus 0)$ by an action of the algebraic group $\Gbb_m \x \Gbb_m$. In coordinates, the action is given by
\[
(\lambda, \mu) \cdot (t_1, t_2, x_1, \dots, x_n) =
(\lambda t_1, \lambda t_2, \mu \lambda^{-a_1} x_1, \dots, \mu \lambda^{-a_n} x_n).
\]
For more details on both points of view, see \cite[Chapter 2]{Reid-surfaces}.


%%% Local Variables: 
%%% mode: latex
%%% TeX-master: "ag-examples"
%%% End: 
